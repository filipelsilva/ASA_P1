\documentclass[a4paper, 12pt]{article}
\headheight=75pt
\usepackage[T1]{fontenc}
\usepackage[utf8]{inputenc}
\usepackage[portuguese]{babel}
\usepackage[margin=3.0cm, top=5.0cm, headsep=0.5cm]{geometry}
\usepackage{fancyhdr}

\author{Filipe Ligeiro Silva (95574) e Ricardo Filipe Santos Martins (95662)}
\title{Relatório do Projeto 1}
\date{}

\pagestyle{fancyplain}
\fancyhf{}
\fancyhead[C]{\LARGE{Relatório 1º Projeto ASA 2020/2021}\\~\\}
\fancyhead[L]{\large{\textbf{Grupo:} al084\\\textbf{Alunos:} Filipe Silva (95574) e Ricardo Martins (95662)}}

\begin{document}
\section*{Descrição do Problema e da Solução}
https://www.mathcs.emory.edu/~cheung/Courses/171/Syllabus/11-Graph/Docs/longest-path-in-dag.pdf
\section*{Análise Teórica}
\section*{Avaliação Experimental dos Resultados}
\end{document}

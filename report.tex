\documentclass[a4paper, 12pt]{article}
\headheight=75pt
\setlength{\parindent}{4ex}
\usepackage[T1]{fontenc}
\usepackage[utf8]{inputenc}
\usepackage[portuguese]{babel}
\usepackage[margin=3.0cm, top=5.0cm, headsep=0.5cm]{geometry}
\usepackage{fancyhdr}
\usepackage{indentfirst}
\usepackage{hyperref}

\author{Filipe Ligeiro Silva, Ricardo Filipe Santos Martins}
\title{Relatório 1º Projeto ASA 2020/2021}
\date{}

\pagestyle{fancyplain}
\fancyhf{}
\fancyhead[C]{\large{Relatório 1º Projeto ASA 2020/2021}\\~\\}
\fancyhead[L]{\textbf{Grupo:} al084\\\textbf{Alunos:} Filipe Silva (95574) e Ricardo Martins (95662)}

\begin{document}
\section*{Descrição do Problema e da Solução}
O problema apresentado consiste em, dada uma sequência de dominós, determinar o
menor número de dominós que seria preciso derrubar com a mão para que todos
caíssem, e, nessas condições, qual seria o maior número de dominós que cairiam
seguidos. A sequência de dominós é representada como um \textit{DAG} (Grafo
Dirigido Acíclico).

Após análise do problema, apercebemo-nos de que o número mínimo de dominós a
mandar abaixo seria o número de \textit{sources} presentes no \textit{DAG}.
Para achar o maior número de dominós a cair em sequência, é necessário descobir
as distâncias dos nós a um ponto de origem. Ordenando topologicamente o
\textit{DAG} e fazendo a travessia dos \textit{sources} para os \textit{sinks},
atualizando as distâncias de um nó com base na distância máxima dentre as dos
nós que lhe acedem, é possível determinar a distância máxima, resolvendo assim
o problema.

\textbf{Fontes: }
\url{https://www.mathcs.emory.edu/~cheung/Courses/171/Syllabus/11-Graph/Docs/longest-path-in-dag.pdf}
\section*{Análise Teórica}
\section*{Avaliação Experimental dos Resultados}
\end{document}
